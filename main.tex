\documentclass[twocolumn,floatfix,nofootinbib,aps]{revtex4-1}
\usepackage[utf8]{inputenc}

\usepackage{algorithmicx, algpseudocode, algorithm}
\usepackage{amsmath}    % need for subequations
\usepackage{amssymb}    % for symbols
\usepackage{graphicx}   % need for figures
\usepackage{verbatim}   % useful for program listings
\usepackage{color}      % use if color is used in text
\usepackage{subfigure}  % use for side-by-side figures
%\usepackage{hyperref}   % use for hypertext links, including those to external documents and URLs

% new commands
\DeclareMathOperator*{\argmax}{argmax\,}
\renewcommand{\vert}{\, | \,}

\usepackage[capitalise]{cleveref}   % use for referencing figures/equations
\begin{document}
\title{Notes on Simulating Mutants: Efficient MSM-based Sampling Strategies}
\author{Robert T. McGibbon}

\begin{abstract}
\end{abstract}
\maketitle

\section{Introduction}
Mutational analysis is one of the central tools in experimental protein science, but remains a challenge for simulation. Using standard simulation
approaches, the study of a protein and a single mutant requires 2x the computational effort of studying just the original protein. Can we use MSMs to do better?

Assume that we've run extensive, converged simulation of a protein, $A$, and we build an MSM. We now want to run a mutant, $A'$. We assume that $A$, and $A'$ share a common state space -- the interest is in how the mutation affects
the transition probabilities.

If we assume that the mutation affects the transition matrix in a ``small'' way -- that the mutation is appropriately classified as a perturbation, then it should be possible to ``win''. First, short trajectories (of length equal to a single lag-time) are sufficient, because (by assumption), we don't have to discover any new states. We just have to estimate the perturbed transition probabilities. Second, because the mutation is small, there's a lot of mutual information between transition probabilities in $A$ and those in $A'$. We're not starting from scratch here.

\section{Problem Statement}
What is the mathematical model for the cross-talk between $A$ and $A'$? Our observation of $A$ must basically be the prior for $A'$.

Consider a row of the transition matrix $T^{A'}$, the transition probabilities
leaving state $i$, $\vec{p}_i^{A'}$. In the absence of any simulation data on $A'$, what is our prior distribution on $P(\mathbf{p_i}^{A'})$?

The simplest idea is that the $\vec{p}_i^{A'}$ are $\operatorname{Dir}(\mathbf{\alpha})$, where the $\mathbf{\alpha}$ parameters (effectively the psuedocounts) are determined by the number of observed counts in protein $A$, $\vec{c}_i^A$. If there is a lot of mutual information between $A$ and $A'$, then the $\alpha$ should be equal to $\vec{c}_i^A$. This formalizes the idea that our predictions about protein $A'$ made from data collected on $A$ are equally confident as our predictions about protein $A$ using the same data collected on $A$. But if there's not a lot of mutual information between the two proteins, then our prior on $\vec{p}_i^{A'}$ should be non-informative.

$$
\vec{p}_i^{A'} \sim \operatorname{Dir}(q_i \cdot \vec{c}_i^A + 1/2)
$$

Here, the parameter $q_i \in [0,1]$ gives something like the expected strength of the information transfer between state $i$ in the the two models. When $q_i=1$, a count measured in $A$ is ``worth it's weight'' in the $A'$ model. But as $q_i \rightarrow 0$, those counts are worthless in $A'$ and we get a noninformative Jeffreys prior.

This statement of the problems is nice because it takes into account some uncertainty in the gold-standard model for $A$. It doesn't just look the MLE estimate of the transition matrix in $A$, but uses the counts directly to parameterize the distribution over $A'$. So for regions of the state space where $\vec{c}_i^{A'}$ is low, we're going to get a mostly uninformative prior naturally.

One question: should $q$ be a single parameter for the whole model, or should every state have its own $q_i$? The later case could encode the idea that some of the states behave very similarly in $A$ vs. $A'$, whereas other might be very different.

Next: clearly $q_i$ should not be a fixed parameter. It should also be a random variable, estimated from the data. Perhaps the appropriate prior on $q_i$ is $q_i \sim \operatorname{Beta}(\alpha, \beta)$.

Now, if observe some outbound counts, $\vec{K}$, from state $i$ in simulations of the mutant protein $A'$ (note for consistency, we could notate $\vec{K}$ as $\vec{c}_i^{A'}$ instead), they are distributed as a multinomial with parameters $\vec{p}_i^{A'}$. The model specified is then:
 
\begin{align*}
    \vec{c}_i^A &= \text{prior observed counts in protein $A$}\\
    \alpha, \beta &= \text{$q$'s hyperparameters}\\
    q_i &\sim \operatorname{Beta}(\alpha, \beta) \\
    \vec{p}_i^{A'} &\sim \operatorname{Dirichlet}(q \cdot \vec{c}_i^A + 1/2)\\
    K &\sim \operatorname{Multinomial}(\vec{p}_i^{A'})
\end{align*}

Can we do inference here? I think so. Let's start with the joint distribution of of $\vec{p}_i^{A'}$ and $q_i$. By bayes rule,

$$
P(q_i, \vec{p}_i^{A'} \vert \vec{K}) \propto P(\vec{K} \vert q, \vec{p}_i^{A'}) \;  P(q_i, \vec{p}_i^{A'})
$$

$K$ is conditionally independent of $q_i$ given $\vec{p}_i^{A'}$, and is multinomial. $P(q_i, \vec{p}_i^{A'})$ factors as $P(\vec{p}_i^{A'} \vert q_i) \cdot  P(q_i)$, a Dirichlet times a Beta. Because of the conjugacy, $P(\vec{K} \vert q_i, \vec{p}_i^{A'})$ and $P(\vec{p}_i^{A'} \vert q_i)$ group together into an updated Dirichlet. Putting it all together, we get

$$
P(q_i, \vec{p}_i^{A'} \vert \vec{K}) \propto \operatorname{Dir}(q_i \cdot \vec{c}_i^{A} + \vec{K} + 1/2) \cdot P_{\alpha, \beta}(q)
$$

And then marginalizing out $q_i$, we have


$$
P(\vec{p}_i^{A'} \vert \vec{K}) \propto \int_0^1 dq \; \operatorname{Dir}(q \cdot \vec{c}_i^{A} + \vec{K} + 1/2) \cdot  P_{\alpha, \beta}(q_i)
$$

With MCMC, I think sampling from this posterior is pretty straightforward. But what about an analytic solution? When I write out the integral explicitly, it looks pretty bad.

\begin{align*}
P(\vec{p}_i^{A'} &= \vec{x} \vert \vec{K}) \propto \int_0^1 dq \; \frac{1}{Z(q_i)} \prod_{j=1} x_j^{q_i \cdot \vec{c}_{ij}^{A'} + \vec{K}_j + 1/2} q_i^{\alpha-1}(1-q_i)^{\beta-1}\\
Z(q_i) &= \frac{1}{B(q \cdot \vec{c}_i^{A} + \vec{K} + 1/2) \cdot B(\alpha, \beta)}
\end{align*} 

But if instead we knew $q$ directly -- i.e. by just fixing at a certain value, then the distribution for $P(\vec{p}_i^{A'})$ would be simple. Maybe we do MCMC to sample over $q$, and then use analytic formulas given $q$...

\section{Active Learning}
We've done a little bit of simulation in $A'$, and now we want to start some new simulations. From which state $i$ should we start our next round of sampling?

We should choose a state $i$ to simulate from that maximizes the expected K-L divergence between the revised posterior (after observing a new count) and the current posterior distribution based on the data we already have. This is basically choosing the state that maximizes the expected information gain.

The K-L divergence from one Dirichlet distribution, $p$, to another, $q$, is given \url{www.fil.ion.ucl.ac.uk/~wpenny/publications/densities.ps} as

\begin{align*}
D_{KL}(\lambda_q || \lambda_p) = \log& \frac{\Gamma(\lambda_{qt})}{\Gamma(\lambda_{pt})} + \sum_{s=1}^m \log \frac{\Gamma(\lambda_p(s))}{\Gamma(\lambda_q(s))} \\
&+ \sum_{s=1}^m \left[\lambda_q(s) -\lambda_p(s)\right]\left[\Psi(\lambda_q(s)) - \Psi(\lambda_{qt})\right]
\end{align*}

Where $\lambda_p$ and $\lambda_q$ are the parameters of the two Dirichlet distributions, $\lambda_{qt} = \sum_{s=1}^{m}\lambda_q(s)$, $\lambda_{pt} = \sum_{s=1}^{m}\lambda_p(s)$, and $m$ is the number of states.

But how about the expected K-L divergence after collecting an additional sample, $L$?

$$
E[D_{KL}(\lambda_q || \lambda_p + \vec{L})] = ?
$$

We going to have to sample over $\vec{L}$. Also, this analytic form for the K-L divergence is only right conditional on $q_i$, because it's only when conditioned on $q_i$ that we have a Dirichlet distribution for $\vec{p}_i^{A'}$.

So here's the procedure: first, we run an MCMC sampler over this Beta-Dirichlet-Multinomial model, keeping track of the posterior distribution samples over $\vec{p}_i^{A'}$ and $\alpha = q_i \cdot \vec{c}_i^A + \vec{K} + 1/2$. For each of these samples, we simulate draws from the multinomial, make ``virtual'' updates to the model at fixed $q$, and check the K-L divergence.\\



\begin{algorithm}[H]
\centering
\begin{algorithmic}[c]
\State Sampled-KLs $\gets$ List()
\For {$(\vec{p}_i^{A'}, \alpha)$ in MCMC posterior samples}
    \For {$i$ in $1..M$}
       \State Sample $L$ from $\operatorname{Mult}(1, \vec{p}_i^{A'})$
       \State Append $D_{KL}(\alpha \,||\, \alpha + L)$ to Sampled-KLs
    \EndFor
\EndFor
\State \Return \textsc{Mean}(Sampled-KLs)
\end{algorithmic}
\caption{MCMC Expected Information Gain}
\end{algorithm}

Note: there is a simplifying assumption built in here. It's that the additional sample $L$ doesn't change $q_i$. I think this is okay, and it's \emph{key} for the tractability because it lets us use the Dirichlet/multinomial conjugacy to update the $\alpha$ with the virtual count.

Then choose the state that maximized this expected information gain.
\end{document}